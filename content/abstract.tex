The demand for compact and high-performance focusing lenses operating at millimeter-scale wavelengths around 60GHz frequencies has spurred innovative approaches to lens design. However, regular devices in this wavelength regime are bulky or may suffer from poor performance. Inverse design methodologies are a paradigm shift in lens engineering. Formulating the required functionality (objective function) and algorithmically looking for the desired material topology is an inversion of the traditional approach. Using inverse-designed topologies allows exploiting a material's capabilities beyond its usual application area and necessary dimensions. This paper presents an inverse-designed lens operating in the near-field of a 60 GHz patch antenna. With a size of around 15x15x15mm compared to their regular counterparts, the presented structure promises enhanced performance by maintaining a reduced form factors. 
